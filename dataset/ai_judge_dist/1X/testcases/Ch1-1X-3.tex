 \documentclass{article}
\usepackage{graphicx} % Required for inserting images
\usepackage{amsmath}
\usepackage{ctex} % 支持中文
\setlength{\parindent}{0pt}  % 取消全局缩进
\title{组合数学: 第一章作业}

\begin{document}

\maketitle

\section*{1.1}

\subsection*{(1) 任何两个男生不相邻 ($m \leq n + 1$)}

\textbf{解答:}

首先,排列 $n$ 个女生,有 $n!$ 种排列方式。

女生之间有 $n + 1$ 个空隙,从中选择 $m$ 个空隙来放置男生,有 $\dbinom{n+1}{m}$ 种空隙选择方式。

最后安排 $m$ 个男生的插入顺序,共有 $m!$ 种排列方式。\\

总方案数:
\[
 \boxed{n! \times \dbinom{n+1}{m} \times m!}
\]

\subsection*{(2) $n$ 个女生形成一个整体}

\textbf{解答:}

女生内部的排列有 $n!$ 种方式。

将 $n$ 个女生视为一个整体,加上 $m$ 个男生,共有 $m + 1$ 个元素, 排列这 $m + 1$ 个元素,有 $(m + 1)!$ 种方式。\\

总方案数:

\[
\boxed{(m + 1)! \times n!}
\]

\subsection*{(3) 男生 $A$ 和女生 $B$ 相邻}

\textbf{解答:}

将男生 $A$ 和女生 $B$ 视为一个整体,与剩余的男生、女生共同组成 $m + n - 1$ 个元素, 共有 $(m + n - 1)!$ 种方式。

同时$A$ 和 $B$ 内部有 $2$ 种排列方式($AB$ 或 $BA$)。\\

总方案数:

\[
\boxed{(m + n - 1)! \times 2}
\]

\section*{1.2}

\subsection*{(1) 任何两个女生不相邻 }

\textbf{解答:}

首先安排男生的位置,$6$ 个男生有 $6!$ 种排列方式, 但考虑圆桌排列需除去6, 最终得到$5!$ 种排列方式。

男生之间形成了 $6$ 个空隙,可以放置 $5$ 个女生, 首先取 $5$ 个空隙出来共 $6$ 种取法, 再排列女生共$5!$ 种排列方式。

因此总方案数:

\[
 \boxed{5! \times 6 \times 5! = 86400}
\]

\subsection*{(2) 所有女生形成一个整体}

\textbf{解答:}

女生内部有 $5!$ 种排列方式。

将 $5$ 个女生视为一个整体,加上 $6$ 个男生,共有 $7$ 个元素, 进行圆桌排列排共有 $6!$ 种方式。

总方案数:

\[
\boxed{6! \times 5! = 86400}
\]

\subsection*{(3) 女生 $A$ 两侧均是男生}

\textbf{解答:}

首因为 $A$ 的两侧必须是男生,所以将 $A$ 和其左右的男生视为一个整体, 从6个男生中抽出两个安排在$A$两侧, 用有$\dbinom{6}{2} \times 2$ = 30 种方法。

加上剩下的 $4$ 个女生和 $4$ 个男生,共 $9$ 个元素。

对 $9$ 个元素进行圆桌排列,共有 $8!$ 种方式。

总方案数:

\[
\boxed{30 \times 8! = 1209600}
\]

\section*{1.3}
\textbf{解答:}

因为:

\[
k \times k! = (k + 1)! - k!
\]

因此,原式可变为:

\[
\sum_{k=1}^{n} k \times k! = \sum_{k=1}^{n} \left( (k+1)! - k! \right) = \sum_{k=1}^{n} (k+1)! - \sum_{k=1}^{n} k!
\]

又:

\[
\sum_{k=1}^{n} (k+1)! = \sum_{k=2}^{n+1} k! = (2! + 3! + \cdots + (n+1)!)
\]

且:

\[
\sum_{k=1}^{n} k! = (1! + 2! + \cdots + n!)
\]

因此,

\[
\sum_{k=1}^{n} k \times k! = \left( (2! + 3! + \cdots + (n+1)!) - (1! + 2! + \cdots + n!) \right) = (n+1)! - 1
\]

计算得到结果为:

\[
\boxed{(n+1)! - 1}
\]

\section*{1.4}
\textbf{解答:}

1. 分解质因数:

\begin{itemize}
    \item $10 = 2 \times 5$,所以 $10^{40} = (2 \times 5)^{40} = 2^{40} \times 5^{40}$
    \item $20 = 2^2 \times 5$,所以 $20^{30} = (2^2 \times 5)^{30} = 2^{60} \times 5^{30}$
\end{itemize}

2. 求最大公因数:

\begin{itemize}
    \item 对于质因数 $2$:最小指数为 $\min(40, 60) = 40$
    \item 对于质因数 $5$:最小指数为 $\min(40, 30) = 30$
\end{itemize}

因此,

\[
\text{GCD}(10^{40}, 20^{30}) = 2^{40} \times 5^{30}
\]

3. 计算公因数的个数:

因为所有公因数都是最大公因数的因数,那么由最大公因数可以求得所有公因数的数目。$2^{40}$ 有 41 种取法($2^0, 2^1, 2^2, \cdots, 2^{40}$),相应的,$5^{30}$ 有 31 种取法,那么公因数的总数目为

\[
41 \times 31 = 1271
\]

计算得到结果为:

\[
\boxed{1271}
\]

\section*{1.5}

\textbf{解答:}

考虑从 $000000$ 到 $999999$ 这$1000000$个数共 $1000000 \times 6 = 6000000 $ 位数 \\

\text{易见数字0、1、2…9出现的次数相等,都是} $ 6000000 \div 10 = 600000 $ 次 \\

\text{因此从1到999999中数字1到9出现共} $600000 \times 9 = 5400000$ 个 \\

\text{又从1到999999,共有} \\
$9 \times 1 + 90 \times 2 + 900 \times 3 + 9000 \times 4 + 90000 \times 5 + 900000 \times 6 \\
= 9 + 180 + 2700 + 36000 + 450000 + 5400000 \\
= 5888889 $ 位数 \\

\text{因此从1到999999,数字0出现} $5888889 - 5400000 = 488889$ 次 \\

最后加上1000000中的6个, 计算得到结果为:

\[
\boxed{488895}
\]

\section*{1.6}
\textbf{解答:}

使用“隔板法”, 将问题转换为用 $r - 1$ 个隔板将 $n$ 个球分进 $r$ 个隔间。

计算得到结果为:

\[
 \boxed{\binom{n - 1}{r - 1}}
\]

\section*{1.7}
\textbf{解答:}

 \text{确保每个盒子至少有} \( k \) \text{个球:} \\
由于每个盒子中至少有 \( k \) 个球,可以先给每个盒子分配 \( k \) 个球。这样一来,已经分配了 \( r \times k \) 个小球。

\text{剩余球的分配:} \\
现在剩下的球数是 \( n - r \times k \) 个球。这些球可以任意分配到 \( r \) 个盒子中,每个盒子可以有 0 个或多个剩余的球。

\text{将剩余的球分配到} \( r \) \text{个盒子中:} \\
通过“隔板法”,方案数为从 \( (n - r \times k) + (r - 1) \) 个位置中选 \( r - 1 \) 个位置放隔板:
\[
\binom{n - r \times k + r - 1}{r - 1}
\]

计算得到结果为:

\[
\boxed{\binom{n - (k - 1) \times r - 1}{r - 1}}
\]

\section*{1.8}
\textbf{解答:}

先分配小球进盒子中, 再将剩余的空盒子插入非空盒子之间的空隙中。\\

\textbf{情况1: 每个盒子最多装一个球, 5个小球各处于一个盒子内} \\

$k = 3$ (空盒子数),$m = 5$ (非空盒子数)

$\bullet$ 空盒子位置安排数量: $C(6, 3) = 20$。

$\bullet$ 5个小球全排列: $5! = 120$。

$\bullet$ 总方案数: $20 \times 120 = 2400$。\\

\textbf{情况2: 要求非空盒子不相邻至少要让4个盒子非空, 其中一个盒子装了2个小球} \\

$k = 4$,$m = 4$

$\bullet$ 空盒子位置安排数量: $C(5, 4) = 5$。

$\bullet$ 从5个小球中选出2个当成一组, 4组全排列: $4! \times C(5, 2) = 24 \times 10 = 240$。

$\bullet$ 总方案数: $5 \times 240 = 1200$。\\


\textbf{总计: } $2400 + 1200 = 3600$。\\

计算得到结果为:

\[
\boxed{3600}
\]

\section*{1.9}
\textbf{解答:}

\textbf{(1):}

集合 \( A \) 表示横坐标在 \( [0,9] \) 和纵坐标在 \( [0,5] \) 的平面格点。  \\

- 选择横坐标的方式:从 10 个点中选择 2 个点,方式有 \( C(10, 2) = \frac{10 \times 9}{2} = 45 \) 种。

- 选择纵坐标的方式:从 6 个点中选择 2 个点,方式有 \( C(6, 2) = \frac{6 \times 5}{2} = 15 \) 种。

因此,长方形的总数为 \( 45 \times 15 = 675 \) 个。\\

计算得到结果为:

\[
\boxed{675}
\] \\

\textbf{(2):}

为了构成一个与坐标轴平行的正方形, 选定左下角横纵坐标后, 其他三个角的作坐标便已固定。为了确保这个正方形的右边横坐标 $a + k$ 不超过最大横坐标值(9),需要满足:
\[
a + k \leq 9
\]
因此,左下角横坐标$a$ 最大只能取到 $9 - k$,也就是说,横坐标 $a$ 的取值范围是从 $0$ 到 $9 - k$,这意味着共有 $10 - k$ 种可能的横坐标组选择, 对于纵坐标也同理。\\

分别计算每种边长对应的正方形数量:\\

- 边长为 1 时:可以选择的左下角横坐标有 \( 10 - 1 = 9 \) 个,左下角纵坐标有 \( 6 - 1 = 5 \) 个,正方形数量为 \( 9 \times 5 = 45 \)。\\

- 边长为 2 时:可以选择的左下角横坐标有 \( 10 - 2 = 8 \) 个,左下角纵坐标有 \( 6 - 2 = 4 \) 个,正方形数量为 \( 8 \times 4 = 32 \)。\\

- 边长为 3 时:可以选择的左下角横坐标有 \( 10 - 3 = 7 \) 个,左下角纵坐标有 \( 6 - 3 = 3 \) 个,正方形数量为 \( 7 \times 3 = 21 \)。\\

- 边长为 4 时:可以选择的左下角横坐标有 \( 10 - 4 = 6 \) 个,左下角纵坐标有 \( 6 - 4 = 2 \) 个,正方形数量为 \( 6 \times 2 = 12 \)。\\

- 边长为 5 时:可以选择的左下角横坐标有 \( 10 - 5 = 5 \) 个,左下角纵坐标有 \( 6 - 5 = 1 \) 个,正方形数量为 \( 5 \times 1 = 5 \)。\\

因此,正方形的总数为 \( 45 + 32 + 21 + 12 + 5 = 115 \) 个。

计算得到结果为:

\[
\boxed{115}
\] \\

\section*{1.10}
\textbf{解答:}\\
\textbf{(1)}\\
设选择了 $k$ 个 0,其中 $0 \leq k \leq n$。从元素 1 到 $n$ 中选取剩余的 $n - k$ 个元素。\\

\begin{itemize}
    \item 从 $n$ 个 0 中选择 $k$ 个 0 的方式只有 1 种(因为所有 0 是相同的)。
    \item 从 $n$ 个不同的元素中选取 $n - k$ 个元素的方案数为:
    \[
    C(n, n - k) = C(n, k)
    \]
    \item 总方案数为(二项式定理):
    \[
    \sum_{k=0}^{n} C(n, k) = 2^n 
    \]
\end{itemize}

计算得到结果为:

\[
\boxed{2^n}
\] \\

\textbf{问题 (2):}\\
同第一问

\begin{itemize}
    \item 从 $n$ 个 0 选择 $k$ 个 0 的方式只有 1 种。
    \item 从 $2n + 1$ 个不同的元素中选取 $n - k$ 个元素的方案数为:
    \[
    C(2n + 1, n - k)
    \]
    \item 总方案数为(由组合数的对称性及二项式定理有):
    \[
    \sum_{k=0}^{n} C(2n + 1, n - k) = \sum_{k=0}^{n} C(2n + 1, k) = \frac{1}{2}\sum_{k=0}^{2n+1} C(2n+1, k) \cdot 1^k = 2^{2n}
    \]
    
\end{itemize}

计算得到结果为:

\[
\boxed{2^{2n}}
\] \\

\section*{1.11}

\textbf{解答: }

\begin{itemize}
    \item $m$ 为学生总人数。
    \item $k$ 为分配到机器 1 的学生人数,因此机器 2 也有 $k$ 名学生。
    \item 剩余的 $m - 2k$ 名学生分配到机器 3、4、5。
\end{itemize}

\( k \) 的取值范围为:
\[
0 \leq k \leq \left\lfloor \frac{m}{2} \right\rfloor
\]
\\
\textbf{情况一 : 假设不区分学生个体, 单纯只考虑每个机器分配多少个学生}\\

转化为“隔板法”, 分配方式的数量为:
\[
\binom{(m - 2k) + 2}{2} = \binom{m - 2k + 2}{2}
\]\\

总方案数就是遍历所有可能的 \( k \) 值,对每个 \( k \) 的情况计算出相应的分配方案数并累加。  \\

计算得到结果为:
\[
\boxed{\sum_{k=0}^{\left\lfloor \frac{m}{2} \right\rfloor} \binom{m - 2k + 2}{2}}
\]
\\
\textbf{情况二 : 区分学生个体}\\

由于每个学生必须被分配到一台机器上,且学生是可区分的,需要计算以下内容:

\subsection*{第一步:选择分配到机器 1 和机器 2 的学生}

从 $m$ 名学生中选择 $2k$ 名学生,分配到机器 1 和机器 2 上。

选择方法数为:
\[
C(m, 2k)
\]

\subsection*{第二步:将选出的 $2k$ 名学生划分为两组}

将 $2k$ 名学生划分为两个大小为 $k$ 的组,分别分配到机器 1 和机器 2。

划分方法数为:
\[
C(2k, k)
\]

这是因为在 $2k$ 名学生中选择 $k$ 名学生分配到机器 1,剩下的 $k$ 名学生分配到机器 2。

\subsection*{第三步:分配剩余的学生}

剩余的 $m - 2k$ 名学生可以任意分配到机器 3、4、5。

每个学生有 3 种选择(机器 3、4 或 5)。

分配方法数为:
\[
3^{m - 2k}
\]

\subsection*{第四步:综合计算}

\text{对每个$k$下的方案数都有: } 
\[
C(m, 2k) \times C(2k, k) \times 3^{m - 2k}
\]\\

总方案数为对所有可能的 $k$ 求和, 计算得到结果为:

\[
\boxed{\sum_{k=0}^{\left\lfloor \dfrac{m}{2} \right\rfloor} C(m, 2k) \times C(2k, k) \times 3^{m - 2k}}
\]

\section*{1.12}

\textbf{解答: }
可以将问题转化为如下的路径问题:

\begin{itemize}
    \item 从起点 $(0, 0)$ 出发,路径长度为 $2n$,终点为 $(2n, 0)$。
    \item 向上一步(0): 从点 (x, y) 移动到 (x+1, y+1)。
    \item 向下一步(1): 从点 (x, y) 移动到 (x+1, y-1)。
    \item 要求路径始终不低于水平轴(即 $y \geq 0$)。
\end{itemize}

\subsection*{步骤 1:计算总路径数}
不考虑限制条件,二进制串的总数为:
\[
\binom{2n}{n}
\]

\subsection*{步骤 2:计算不合法路径数}

如果路径在某一点第一次跌到 y = -1,那么将从该点开始的路径沿着 y = -1 水平线反射,得到一条新的路径。这种反射将路径从 (2n, 0) 变为 (2n, -2),也就是说,这些不合法的路径与从 (0, 0) 到 (2n, -2) 的所有路径是一一对应的。
\begin{itemize}
    \item 总步数:$2n$ 步,其中有 $n$ 个向上步(0),$n$ 个向下步(1)。
    
    \item 由于终点在 $y = -2$,向下步比向上步多 2 步:
    \[
    \text{向上步数} = n - 1
    \]
    \[
    \text{向下步数} = n + 1
    \]

    \item 因此,不合法路径的数目为:
    \[
    \text{不合法路径数} = \binom{2n}{n + 1} = \binom{2n}{n - 1}
    \]
\end{itemize}

\subsection*{步骤 3:计算合法路径数}
合法路径数为
\[
\binom{2n}{n} - \binom{2n}{n-1} = \frac{1}{n + 1} \binom{2n}{n}
\]\\

因此,满足条件的二进制串数目的计算结果为:
\[
\boxed{\frac{1}{n + 1} \binom{2n}{n}}
\]

\end{document}



