% Homework template for Inference and Information
% UPDATE: September 26, 2017 by Xiangxiang
\documentclass[a4paper]{article}
\usepackage{ctex}
\ctexset{
proofname = \heiti{证明}
}
\usepackage{amsmath, amssymb, amsthm}
% amsmath: equation*, amssymb: mathbb, amsthm: proof
\usepackage{moreenum}
\usepackage{mathtools}
\usepackage{url}
\usepackage{bm}
\usepackage{enumitem}
\usepackage{graphicx}
\usepackage{subcaption}
\usepackage{booktabs} % toprule
\usepackage[mathcal]{eucal}
\usepackage[thehwcnt = 1]{iidef}
\usepackage[backend=bibtex]{biblatex}


\thecourseinstitute{清华大学}
\thecoursename{组合数学}
\theterm{2024年秋季学期}
\hwname{作业}
\slname{\heiti{解}}
\begin{document}
\courseheader


\begin{enumerate}
  \setlength{\itemsep}{3\parskip}
%第一题
  \item 设有 $m$ 个男生和 $n$ 个女生排成一行($m, n$ 均为正整数),若
  \begin{enumerate}
      \item 任何两个男生不相邻(即 $m \leq n + 1$);
      \item $n$ 个女生形成一个整体(即任何两个女生之间没有男生);
      \item 男生 $A$ 和女生 $B$ 相邻。
  \end{enumerate}

\begin{solution}
  \begin{enumerate}
    \item 先把$n$个女生排成一列,有$P(n,n)$种方案,女生列中一共有$n+1$个空格可以插入,再把$m$个男生插入排好的女生列中,有$P(n+1,m)$种方案,
          所以总方案数为:\\
          \begin{center}
            $P(n,n)\times P(n+1,m)$ 
          \end{center}
          
    \item 先把$n$个女生捆绑排序,有$P(n,n)$种方案,再将捆绑后的女生作为一个整体插入$m$个排好男生中,有$P(m,m)\times C(m+1,1)$种方案,
          所以总方案数为:\\
          \begin{center}
            $P(n,n)\times P(m,m)\times C(m+1,1)$ 
          \end{center}
          
    \item 先将同学$AB$捆绑,有2种可能,作为一个整体与其他同学进行排序,一共$m+n-1$个进行排序,有$P(m+n-1,m+n-1)$种方案,
          所以总方案数为:
          \begin{center}
            $P(m+n-1,m+n-1)\times 2$ 
          \end{center}
             
\end{enumerate}
\end{solution}
%第二题
\item 6个男生和5个女生围在一圆桌旁, 若
\begin{enumerate}
      \item 任何两个女生不相邻;
      \item 所有女生形成一个整体;
      \item 女生 A两侧均是男生.
      分别讨论有多少种方案
\end{enumerate}

\begin{solution}
      \begin{enumerate}
        \item 先把$6$个男生围成一圈,相当于圆排列,有$\frac{P(6,6)}{6}$种方案,为了达到任意两个女生不相邻,
              将女生插入男生的空格中,对于每一种男生圆排列,一共有六个空格,所以此时女生可以插入的排列数为$P(6,5)$种,
              所以总方案数为:\\
              \begin{center}
                  $\frac{P(6,6)}{6}\times P(6,5)=\frac{6!}{6}\times \frac{6!}{1!}=86400$ 
              \end{center}
              
        \item 先把$5$个女生捆绑当成一个整体,有$P(5,5)$种,再与$6$个男生一起进行圆排列,有$\frac{P(7,7)}{7}$种方案,
              所以总方案数为:\\
              \begin{center}
            $P(5,5)\times \frac{P(7,7)}{7}=86400$ 
              \end{center}
        \item 先将同学$A$与其左右的两个男生捆绑,先从$6$个男生选两个且存在顺序,有$C(6,2)\times P(2,2)$种可能,3个人作为一个整体与其他8个同学进行圆排序,一共$\frac{P(9,9)}{9}$种,
              所以总方案数为:\\
              \begin{center}
            $C(6,2)\times P(2,2)\times \frac{P(9,9)}{9}=1209600$
              \end{center}
    \end{enumerate}
\end{solution}

%第三题
\item 计算$\sum_{k=1}^nk\cdot k!=1\cdot1!+2\cdot2!+\cdots+n\cdot n!$ 
\begin{solution}
      \begin{enumerate}
            猜测原式等于$\sum_{k=1}^nk\cdot k!=(n+1)!-1$,用数学归纳法证明:\\
            (1)归纳奠基:\\当n=1时得,$\sum_{k=1}^1k\cdot k!=1=(1+1)!-1$命题成立;\\ 
            (2)归纳假设:\\假设当n=k时命题成立,即$\sum_{k=1}^nk\cdot k!=(n+1)!-1$;\\ 
            (3)归纳递推:\\
            当n=k+1时,原式为:
            \begin{align*}
                  \sum_{k=1}^{n+1}k\cdot k!&=\sum_{k=1}^nk\cdot k!+(n+1)\cdot (n+1)! \\
                  &=(n+1)!-1+(n+1)\cdot (n+1)!\\
                  &=(n+2)\cdot (n+1)!-1\\
                  &=(n+2)!-1
            \end{align*}
            即命题也成立\\
            所以原式$\sum_{k=1}^nk\cdot k!=(n+1)!-1$
    \end{enumerate}
\end{solution}

%第四题
\item 求$10^{40}$与$20^{30}$的公因数的数目
\begin{solution}
      \begin{enumerate}
      将$10^{40}$与$20^{30}$因式分解:
      $10^{40}=2^{40}\cdot 5^{40},20^{30}=2^{60}\cdot 5^{30}$,
      \\所以最大公因数为$2^{40}\cdot 5^{30}$,
      即原命题可以等价于在40个2中选几个2,在30个5中选几个5相乘组成的公因式的数目,而选择几个2有41种选法,选择几个5有31种选法,
      所以公因数的数目一共有$31*41=1271$个。
     \end{enumerate}
\end{solution}

%第五题
\item 求从1到1000000的整数的十进制表示中,数字0出现的总次数
\begin{solution}
      \begin{enumerate}
      (1)当整数为1位时,数字0出现的次数为0;\\
      (2)当整数为2位时,数字0只能在个位,出现的次数为9;\\
      (3)当整数为3位时,数字0出现在个、十位,百位为$1\sim9$,出现的次数为:\\
      \begin{center}
            $9\cdot (C(2,1)\cdot 9+1\cdot 2)$\\
      \end{center}
      (4)当整数为4位时,数字0出现在个、十和百位,千位为$1\sim9$,出现的次数为:\\
      \begin{center}
             $9\cdot (C(3,1)\cdot 9\cdot 9+C(3,2)\cdot 9\cdot 2+1\cdot 3)$\\
      \end{center}
      (5)当整数为5位时,数字0出现在个、十、百和千位,万位为$1\sim9$,出现的次数为:\\
      \begin{center}
            $9\cdot (C(4,1)\cdot 9^3+C(4,2)\cdot 9^2\cdot 2+C(4,3)\cdot 9\cdot 3+1\cdot 4)$\\
      \end{center}
      (6)当整数为6位时,数字0出现在个、十、百、千和万位,十万位为$1\sim9$,出现的次数为:\\
      \begin{center}
            $9\cdot (C(5,1)\cdot 9^4+C(5,2)\cdot 9^3\cdot 2+C(5,3)\cdot 9^2\cdot 3+C(5,4)\cdot 9\cdot 4+1\cdot 5)$\\
      \end{center}
      (7)当整数为7位时,数字0出现的次数为6;\\   
      所以对应数字0出现的总次数为情况$(1)\sim(7)$的累加和,即488895;  
\end{enumerate}
\end{solution}

%第六题
\item 将 $n$ 个相同的小球放入 $r$ 个不同的盒子中 ($n \geq r$),禁止出现空盒,求方案数。
\begin{solution}
      \begin{enumerate}
      为了防止空盒,先给r个盒子中各放一个,将剩下的$n-r$个球和$r-1$隔板一起排列,所以方案数为:\\
      \begin{center}
          $C(n-r+r-1,r-1)=C(n-1,r-1)$    
      \end{center}
      
\end{enumerate}
\end{solution}

%第七题
\item 将 $n$ 个相同的小球放入 $r$ 个不同的盒子中,每个盒子中至少有 $k$ 个球($n \geq rk$),求方案数。
\begin{solution}
      \begin{enumerate}
      为了保证每个盒子至少有k个球,先给r个盒子中各放k个,将剩下的$n-rk$个球和$r-1$隔板一起排列,所以方案数为:\\
      \begin{center}
           $C(n-rk+r-1,r-1)$   
      \end{center}
      
\end{enumerate}
\end{solution}

%第八题
\item 8 个盒子排成一列,将5个不同的小球放入这些盒子,要求空盒不相邻,求方案数。
\begin{solution}
      \begin{enumerate}
      为了保证空盒不相邻可以分为两种情况,5个小球分别放入四个盒子和五个盒子:\\
      (1)五个小球放入四个盒子:一个盒子放2个球,其他三个盒子放1个球,将其余的四个盒子插入其中,方案数为:\\
      \begin{center}
            $C(5,2)\cdot P(4,4)\cdot C(5,4)=10\cdot 4!\cdot 5=1200$\\
      \end{center}
      
      (2)五个小球放入五个盒子: 五个小球进行排列后,将其余的三个盒子插入其中,方案数为:\\
      \begin{center}
         $P(5,5)\cdot C(6,3)=5!\cdot 20=2400$ \\   
      \end{center}
      
      总共的方案数为1200+2400=3600种。
\end{enumerate}
\end{solution}

%第九题
\item 设 $A = \{(a,b) \mid a, b \in \mathbb{Z}, 0 \leq a \leq 9, 0 \leq b \leq 5\}$
\begin{enumerate}
      \item 求 $xOy$ 平面上以 $A$ 中的点为四个顶点、四边与坐标轴平行的长方形数目;
      \item 求 $xOy$ 平面上以 $A$ 中的点为四个顶点、四边与坐标轴平行的正方形数目。
\end{enumerate}

\begin{solution}
      \begin{enumerate}
      因为a,b均为整数,a的可能取值$0\sim9$,b的可能取值$0\sim5$:\\
      (1)长方形可以由左下右上对角线的两个顶点确定:若左下顶点坐标为$(i,j)$,则可选的右上顶点数为$(9-i)\cdot (5-j)$\\
      所以总方案数为:$\sum_{i=0}^9\sum_{j=0}^5(9-i)\cdot (5-j)=(9+……+1)(5+……+1)=675$\\
      (2)正方形的个数可以使用面积计算\\可能正方形面积为$1,4,9,16,25$\\
      每个面积对应的正方形个数为:$9\cdot 5,8\cdot 4, 7\cdot 3, 6\cdot 2,5\cdot 1$\\
      所以总正方形数目为$45+32+21+12+5=115$。
\end{enumerate}
\end{solution}

%第10题
\item 分别求从如下多重集中选取 $n$ 个元素的方案数:
\begin{enumerate}
      \item 大小为 $2n$ 的多重集 $\{n \cdot 0, 1, 2, \dots, n\}$;
      \item 大小为 $3n + 1$ 的多重集 $\{n \cdot 0, 1, 2, \dots, 2n + 1\}$。
\end{enumerate}

\begin{solution}
      \begin{enumerate}
      (1)原命题相当于从n个0中选i个,从$1\sim n$中选$n-i$个,方案数为$C(n,n-i)$,
      所以总方案数为:\\
      \begin{center}
            $\sum_{i=0}^nC(n,n-i)=(1+1)^n=2^n$\\
      \end{center}
      
      (2)类似的,原命题相当于从$n$个0中选$i$个,从$1\sim 2n+1$中选$n-i$个,方案数为$C(2n+1,n-i)$,
      所以总方案数为:\\
      \begin{center}
            $\sum_{i=0}^{n}C(2n+1,n-i)=\frac{(1+1)^{2n+1}}{2}=2^{2n}$
      \end{center}
    
\end{enumerate}
\end{solution}

%第11题
\item 5台教学机器编号为 1, 2, 3, 4, 5,分配给 $m$ 名学生使用,且使用第 1 台和第 2 台的人数相等,求分配方案数。

\begin{solution}
      \begin{enumerate}
      假设第1台和第2台的使用人数都为$i$,剩下$m-2i$名学生可以随意在编号3,4,5中选择,所以总方案数为:\\
      假设$m$为偶数:
      \begin{center}
            $\sum_{i=0}^{\frac{m}{2}}C(2i,i)\cdot C(m,2i)\cdot 3^{m-2i}$\\
      \end{center}
      假设$m$为奇数:
      \begin{center}
            $\sum_{i=0}^{\frac{m-1}{2}}C(2i,i)\cdot C(m,2i)\cdot 3^{m-2i}$\\
      \end{center}
    
\end{enumerate}
\end{solution}

%第12题
\item 由 $n$ 个 0 和 $n$ 个 1 构成的 $2n$ 位二进制串,要求任意前 $k$ 位中 0 的数目不少于 1 的数目($1 \leq k \leq 2n$),求满足要求的二进制串的数目。

\begin{solution}
      \begin{enumerate}
      原命题可以等价格路模型,从点$(0,0)$出发,沿$x$轴或$y$轴的正方向每步走一个单位,最终走到$(n,n)$,要求走过的路不能穿过$y=x$。
      从点$(0,0)$出发走到$(n,n)$的总方案数为:
      \begin{center}
            $C(2n,n)$\\
       \end{center}
      
      走过的路不能穿过$y=x$等价于走过的路不能接触$y=x+1$;
      将$(0,0)$关于$y=x+1$对称,得到对称点$(-1,1)$。不符合的方案等价于从点$(-1,1)$出发走到$(n,n)$,方案数为:$C(2n,n-1)$
      所以总方案数为:

      \begin{center}
           $C(2n,n)-C(2n,n-1)$\\
      \end{center}

    
      \end{enumerate}
\end{solution}


\end{enumerate}
\end{document}


%%% Local Variables:
%%% mode: late\rvx
%%% TeX-master: t
%%% End:
