\documentclass{article}
\usepackage{graphicx,ctex,amsmath} % Required for inserting images
\usepackage{fullpage}

\title{组合数学 HW1(全排列问题)}

\begin{document}

\maketitle

\section{字典序法}
83674521的中介数是7244221。而2024的康托展开为
\[2\times 6!+4\times 5!+4\times 5!+4\times 4! +3!+2!.\]
因此,83674521之前第 2024 个排列对应的中介数为$7244221-244110=7000111$,其对应的排列为81235674.


\section{递增进位制数法}
83674521的中介数为$(7442221)\uparrow$. 2024的递增进位制数为$(244110)\uparrow$. 因此,83674521之前第2024个排列对应的中介数为$(7442221)\uparrow-(244110)\uparrow=(7153111)\uparrow$,其对应的排列为86253471.


\section{递减进位制数法}
83674521的中介数为$(1222447)\downarrow$. 2024的递增进位制数为$(11010)\downarrow$. 因此,83674521之前第2024个排列对应的中介数为$(1222447)\downarrow-(11010)\downarrow=(1211437)\downarrow$,其对应的排列为83627451.

\section{邻位对换法}
设$b_2b_3b_4b_5b_6b_7b_8$为83674521的中介数. 我们有:
\begin{itemize}
    \item 2方向向左,背向2的方向上1比2小,故$b_2=1$.

    \item $b_2$为奇数,3方向向右,背向3的方向没有比3小的数,故$b_3=0$.

    \item $b_2+b_3$为奇数,4方向向右,背向4的方向比4小的数只有3,故$b_4=1$.

    \item $b_4$为奇数,5方向向右,背向5的方向比5小的数有$3,4$,故$b_5=2$.

    \item $b_4+b_5$为奇数,6方向向右,背向6的方向比6小的数只有3,故$b_6=1$.

    \item $b_6$为奇数,7方向向右,背向7的方向比7小的数有$3,6$,故$b_7=2$.

    \item $b_6+b_7$为奇数,8方向向右,背向8的方向没有比8小的数,故$b_8=0$.
\end{itemize}
因此83674521的中介数为1012120. 2024的递减进位制数为11010. 因此83674521之前第2024个排列对应的中介数$(1012120)\downarrow-(11010)\downarrow=(1001110)\downarrow$,对应的排列为47632518.

\end{document}
