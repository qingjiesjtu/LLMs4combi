\documentclass{article}

\usepackage{CJKutf8}
\usepackage{setspace}


\title{第一章作业-组合}

\begin{document}
\begin{CJK}{UTF8}{gbsn}
\setstretch{1.25}
\date{}


\maketitle

\section*{字典序法}
\subsection*{中介数} 
首先求解83674521的中介数

前缀先于8的排列的个数:$7×7!$

第一位是3,先于83的排列的个数:$2×6!$

第二位是6,先于836的排列的个数:$4×5!$

第三位是7,先于8367的排列的个数:$4×4!$

第四位是4,先于83674的排列的个数:$2×3!$

第五位是5,先于836745的排列的个数:$2×2!$

第六位是2,先于8367452的排列的个数:$1×1!$

故中介数为 7244221

\subsection*{序号} 
序号为$7×7!+2×6!+4×5!+4×4!+2×3!+2×2!+1×1!=37313$
\subsection*{求解} 
83674521之前的第2024个序列的序号为

37313-2024=35289

对应的中介数计算如下:

35289/7! = 7, 35289 \%7! = 9 
 
9/6! = 0, 9\%6! = 9

9/5! = 0, 9\%5! = 9

9/4! = 0, 9\%4! = 9

9/3! = 1, 9\%3! = 3

3/2! = 1, 3\%2! = 1

1/1! = 1, 1\%1! = 0

因此新中介数为7000111,对应的排列为81235674。



\section*{递增进位制数法}
83674521中介数为(7442221)↑,2024可表示为(244110)↑,(7442221)↑−(244110)↑= (7153111)↑。
而(7153111)↑对应的排列为86253471
\section*{递减进位制数法}
83674521中介数为(1222477)↓,2024可表示为(11010)↓,(1222477)↓−(11010)↓= (1211467)↓。
而(1211467)↓对应的排列为87362451
\section*{邻位对换法}
2向左

b2 = 1,3向右

b3 = 0,4向右

b4 = 1,5向右

b5 = 2,6向右

b6 = 1,7向右

b7 = 2,8向右

中介数为(1012120)↓,2024在递减进位法下表示为(11010)↓,(1012120)↓−
(11010)↓ = (1001110)↓。
对应的排列为47632518
\end{CJK}
\end{document}
