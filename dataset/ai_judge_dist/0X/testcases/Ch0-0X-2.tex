%!TEX program = xelatex
\documentclass{article}
\usepackage{amsmath}
\usepackage{amsfonts}
\usepackage[UTF8]{ctex}
\usepackage{enumerate}
% \usepackage{enumitem}
\usepackage[a4paper, margin=1in]{geometry}
\usepackage{lmodern}
\usepackage{pdfpages}

\title{《组合数学》书面作业 \\
       \large 全排列作业}
\begin{document}
\maketitle

\textit{按照课堂上介绍的四种全排列方法,分别求出 $ 83674521 $ 之前第 $ 2024 $ 个排列。}

\begin{enumerate}
    \item 字典序:首先将原排列转化为中介数:计算每位数后面有多少个比它小的数,得到 $ (7244221)\uparrow $。
    作中介数减法,得到 $ (7244221)\uparrow - 2024 = (7244221)\uparrow - (244110)\uparrow = (7000111)\uparrow $。
    再转换为新排列,得到 $ 81235674 $。
    \item 递增进位制数:首先将原排列转化为中介数:计算每个元素后面有多少个比它小的数,得到 $ (7442221)\uparrow $。
    作中介数减法,得到 $ (7442221)\uparrow - 2024 = (7442221)\uparrow - (244110)\uparrow = (7153111)\uparrow $。
    再转换为新排列,得到 $ 86253471 $。
    \item 递减进位制数:首先将原排列转化为中介数:将递增进位制数反转,得到 $ (1222447)\downarrow $。
    作中介数减法,得到 $ (1222447)\downarrow - 2024 = (1222447)\downarrow - (11010)\downarrow = (1211437)\downarrow $。
    再转换为新排列,得到 $ 83627451 $。
    \item 临位对换法:首先将原排列转化为中介数:从 $ 2 $ 向左开始,依次求得每个元素的方向和 $ b_i $,如下所示,得到 $ (1012127)\downarrow $。
    \begin{center}
    \begin{tabular}{cccc}
        \hline
        $ i $ & 决定方向的值 & 方向 & $ b_i $ \\
        \hline
        $ 2 $ & & 左 & $ 1 $ \\
        $ 3 $ & $ b_2 = 1 $ & 右 & $ 0 $ \\
        $ 4 $ & $ b_3 + b_2 = 1 $ & 右 & $ 1 $ \\
        $ 5 $ & $ b_4 = 1 $ & 右 & $ 2 $ \\
        $ 6 $ & $ b_5 + b_4 = 3 $ & 右 & $ 1 $ \\
        $ 7 $ & $ b_6 = 1 $ & 右 & $ 2 $ \\
        $ 8 $ & $ b_7 + b_6 = 3 $ & 右 & $ 0 $ \\
        \hline
    \end{tabular}
    \end{center}
    作中介数减法,得到 $ (1012120)\downarrow - 2024 = (1012120)\downarrow - (11010)\downarrow = (1001110)\downarrow $。
    再转换为新排列,得到 $ 47632518 $。
\end{enumerate}

\end{document}