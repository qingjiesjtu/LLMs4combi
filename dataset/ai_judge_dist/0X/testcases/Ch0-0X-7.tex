% --------------------------------------------------------------
% This is all preamble stuff that you don't have to worry about.
% Head down to where it says "Start here"
% --------------------------------------------------------------
 
\documentclass[12pt]{article}
 
\usepackage[margin=1in]{geometry} 
\usepackage{amsmath,amsthm,amssymb}
\usepackage{CJKutf8}
\usepackage{enumitem}
\usepackage{graphicx}
 
\newcommand{\N}{\mathbb{N}}
\newcommand{\Z}{\mathbb{Z}}
 
\newenvironment{theorem}[2][Theorem]{\begin{trivlist}
\item[\hskip \labelsep {\bfseries #1}\hskip \labelsep {\bfseries #2.}]}{\end{trivlist}}
\newenvironment{lemma}[2][Lemma]{\begin{trivlist}
\item[\hskip \labelsep {\bfseries #1}\hskip \labelsep {\bfseries #2.}]}{\end{trivlist}}
\newenvironment{exercise}[2][Exercise]{\begin{trivlist}
\item[\hskip \labelsep {\bfseries #1}\hskip \labelsep {\bfseries #2.}]}{\end{trivlist}}
\newenvironment{problem}[2][Problem]{\begin{trivlist}
\item[\hskip \labelsep {\bfseries #1}\hskip \labelsep {\bfseries #2.}]}{\end{trivlist}}
\newenvironment{question}[2][Question]{\begin{trivlist}
\item[\hskip \labelsep {\bfseries #1}\hskip \labelsep {\bfseries #2.}]}{\end{trivlist}}
\newenvironment{corollary}[2][Corollary]{\begin{trivlist}
\item[\hskip \labelsep {\bfseries #1}\hskip \labelsep {\bfseries #2.}]}{\end{trivlist}}

\newenvironment{solution}{\begin{proof}[Solution]}{\end{proof}}
 
\begin{document}
\begin{CJK}{UTF8}{gbsn}
 
% --------------------------------------------------------------
%                         Start here
% --------------------------------------------------------------
 
\title{Combinatorics Homework 1}
\maketitle

\noindent\textbf{方法 1: 字典序法}

83674521 前面的排列数为: $7 \times 7! + 2 \times 6! + 4 \times 5! + 4 \times 4! + 2 \times 3! + 2 \times 2! + 1 \times 1!$

则中介数为: $(7244221)\uparrow$

2024 前面的排列数为: $2 \times 6! + 4 \times 5! + 4 \times 4! + 1 \times 3! + 1 \times 2! + 0 \times 0$

则中介数为: $(244110)\uparrow$

故对应的全排列数为: $(7244221)\uparrow - (244110)\uparrow = (7000111)\uparrow = 81235674$

\vspace{10pt}

\noindent\textbf{方法 2: 递增进位制数法}

83674521 的中介数为: $(7442221)\uparrow$

故对应的全排列数为: $(7442221)\uparrow - (244110)\uparrow = (7153111)\uparrow = 82653471$

\vspace{10pt}

\noindent\textbf{方法 3: 递减进位制数法}

83674521 的中介数为: $(7442221)\uparrow = (1222447)\downarrow$

$2024 = 253 \times 8 + 0$

$253 = 36 \times 7 + 1$

$36 = 6 \times 6 + 0$

$6 = 1 \times 5 + 1$

故 $2024 = (11010)\downarrow$

故对应的全排列数为: $(1222447)\downarrow - (11010)\downarrow = (1211437)\downarrow = 85726431$

\vspace{10pt}

\noindent\textbf{方法 4: 邻位对换法}

$P = \overrightarrow{8}\overrightarrow{3}\overrightarrow{6}\overrightarrow{7}\overrightarrow{4}\overrightarrow{5}\overleftarrow{2}1$

2: $\leftarrow$,$b_{2} = 1$

3: $b_{2}$ 为奇 $\rightarrow$,$b_{3} = 0$

4: $b_{2} + b_{3}$ 为奇 $\rightarrow$,$b_{4} = 1$

5: $b_{4}$ 为奇 $\rightarrow$,$b_{5} = 2$

6: $b_{4} + b_{5}$ 为奇 $\rightarrow$,$b_{6} = 1$

7: $b_{6}$ 为奇 $\rightarrow$,$b_{7} = 2$

8: $b_{6} + b_{7}$ 为奇 $\rightarrow$,$b_{8} = 0$

故对应的全排列数为: $(1012120)\downarrow - (11010)\downarrow = (1001110)\downarrow$

对于 $(1001110)\downarrow$:

2: $\leftarrow$,$b_{2} = 1$

3: $b_{2}$ 为奇 $\rightarrow$,$b_{3} = 0$

4: $b_{2} + b_{3}$ 为奇 $\rightarrow$,$b_{4} = 0$

5: $b_{4}$ 为偶 $\leftarrow$,$b_{5} = 1$

6: $b_{4} + b_{5}$ 为奇 $\rightarrow$,$b_{6} = 1$

7: $b_{6}$ 为奇 $\rightarrow$,$b_{7} = 1$

8: $b_{6} + b_{7}$ 为奇 $\rightarrow$,$b_{8} = 0$

方向为 $(\overleftarrow{1}\overrightarrow{0}\overrightarrow{0}\overleftarrow{1}\overrightarrow{1}\overrightarrow{1}\overrightarrow{0})\downarrow$

故对应的新排列数为: $\overrightarrow{8}\overrightarrow{4}\overrightarrow{7}\overrightarrow{6}\overrightarrow{3}\overleftarrow{2}\overleftarrow{5}1$

\vspace{10pt}

% --------------------------------------------------------------
%     You don't have to mess with anything below this line.
% --------------------------------------------------------------

\end{CJK}
\end{document}
