%
% This is a borrowed LaTeX template file for lecture notes for CS267,
% Applications of Parallel Computing, UCBerkeley EECS Department.
% Now being used for CMU's 10725 Fall 2012 Optimization course
% taught by Geoff Gordon and Ryan Tibshirani.  When preparing
% LaTeX notes for this class, please use this template.
%
% To familiarize yourself with this template, the body contains
% some examples of its use.  Look them over.  Then you can
% run LaTeX on this file.  After you have LaTeXed this file then
% you can look over the result either by printing it out with
% dvips or using xdvi. "pdflatex template.tex" should also work.
%

\documentclass[UTF8,oneside]{article}

\usepackage[UTF8,scheme=plain]{ctex}
% \usepackage[AutoFakeBold,AutoFakeSlant]{xeCJK}  % 载入 xeCJK以支持中文,支持伪粗体,伪斜体
\usepackage[margin=1in]{geometry}
\usepackage{amsmath,amsthm,amssymb}
\usepackage{graphicx}
\usepackage{mathtools}
\usepackage{float, bm}

\setlength{\oddsidemargin}{0.25 in}
\setlength{\evensidemargin}{-0.25 in}
\setlength{\topmargin}{-0.6 in}
\setlength{\textwidth}{6.5 in}
\setlength{\textheight}{8.5 in}
\setlength{\headsep}{0.75 in}
\setlength{\parindent}{0 in}
\setlength{\parskip}{0.1 in}

%
% ADD PACKAGES here:
%

\usepackage{amsmath,amsfonts,graphicx}
\usepackage{etoolbox}
\AtBeginEnvironment{proof}{\normalsize}

%
% The following commands set up the lecnum (lecture number)
% counter and make various numbering schemes work relative
% to the lecture number.
%

%
% The following macro is used to generate the header.
%
\newcommand{\lecture}[4]{
   \pagestyle{myheadings}
   \thispagestyle{plain}
   \newpage
   \setcounter{page}{1}
   \noindent
   \begin{center}
   \framebox{
      \vbox{\vspace{2mm}
    \hbox to 6.28in { {\bf Combination Math
	\hfill 2024 Fall} }
       \vspace{4mm}
       \hbox to 6.28in { {\Large \hfill HW #1  \hfill} }
       \vspace{2mm}
       \hbox to 6.28in { {\it Student: #2 \hfill Time: #3} }
      \vspace{2mm}}
   }
   \end{center}
   \markboth{Lecture #1: #2}{Lecture #1: #2}
}
%
% Convention for citations is authors' initials followed by the year.
% For example, to cite a paper by Leighton and Maggs you would type
% \cite{LM89}, and to cite a paper by Strassen you would type \cite{S69}.
% (To avoid bibliography problems, for now we redefine the \cite command.)
% Also commands that create a suitable format for the reference list.
\renewcommand{\cite}[1]{[#1]}
\def\beginrefs{\begin{list}%
        {[\arabic{equation}]}{\usecounter{equation}
         \setlength{\leftmargin}{2.0truecm}\setlength{\labelsep}{0.4truecm}%
         \setlength{\labelwidth}{1.6truecm}}}
\def\endrefs{\end{list}}
\def\bibentry#1{\item[\hbox{[#1]}]}


\begin{document}

\fontsize{12pt}{24pt}

\section{问题} 求出83674521之前第2024个排列.

\section{字典序法}

首先求出83674521的序号的康托展开为:
\[
    7 \times 7! + 2 \times 6! + 4 \times 5! + 4\times 4! + 2\times 3! + 2\times 2! + 1\times 1!
\]
对应2024的康托展开为:
\[
    2 \times 6! + 4\times 5! + 4\times 4! + 1 \times 3! + 1\times 2!
\]
于是得到所求目标的排列的康托展开为:
\[
    7 \times 7! + 1\times 3! + 1\times 2! + 1\times 1!
\]
于是所求目标的中介数是7000111, 根据中介数, 可得对应的排列为81235674.

\section{递增进位制数法}

83674521的中介数为$(7442221)\uparrow$, 计算序号为:
\[
    7 \times 7! + 4\times 6! + 4\times 5! + 2\times 4! + 2\times 3! + 2\times 2! + 1\times 1!
\]
减去2024后目标序号为:
\[
    7 \times 7! + 1\times 6! + 5\times 5! + 3\times 4! + 1\times 3! + 1\times 2! + 1\times 1!
\]
对应新中介数为$(7153111)\uparrow$, 可得对应的排列为86253471.

\section{递减进位制数法}

83674521的中介数为$(1222447)\downarrow$, 减去2024对应$(11010)\downarrow$, 相减后后新中介数为$(1211437)\downarrow$, 可得对应的排列为83627451.

\section{邻位对换法}

设目标中介数为$(b_2b_3b_4b_5b_6b_7b_8)\downarrow$, 初始2的方向向左,即$
    \overset{}{8}\overset{}{3}\overset{}{6}\overset{}{7}\overset{}{4}\overset{}{5}\overset{\leftarrow}{2}\overset{\leftarrow}{1}
$,可得$b_2=1$;

$i=3$, $b_2$为奇数, 所以3的方向向右,得$
    \overset{}{8}\overset{\rightarrow}{3}\overset{}{6}\overset{}{7}\overset{}{4}\overset{}{5}\overset{\leftarrow}{2}\overset{\leftarrow}{1}
$,可得$b_3=0$;

$i=4$, $b_3+b_2$为奇数, 所以4的方向向右,得$
    \overset{}{8}\overset{\rightarrow}{3}\overset{}{6}\overset{}{7}\overset{\rightarrow}{4}\overset{}{5}\overset{\leftarrow}{2}\overset{\leftarrow}{1}
$,可得$b_4=1$;

$i=5$, $b_4$为奇数, 所以5的方向向右,得$
    \overset{}{8}\overset{\rightarrow}{3}\overset{}{6}\overset{}{7}\overset{\rightarrow}{4}\overset{\rightarrow}{5}\overset{\leftarrow}{2}\overset{\leftarrow}{1}
$,可得$b_5=2$;

$i=6$, $b_5+b_4$为奇数, 所以6的方向向右,得$
    \overset{}{8}\overset{\rightarrow}{3}\overset{\rightarrow}{6}\overset{}{7}\overset{\rightarrow}{4}\overset{\rightarrow}{5}\overset{\leftarrow}{2}\overset{\leftarrow}{1}
$,可得$b_6=1$;

$i=7$, $b_6$为奇数, 所以7的方向向右,得$
    \overset{}{8}\overset{\rightarrow}{3}\overset{\rightarrow}{6}\overset{\rightarrow}{7}\overset{\rightarrow}{4}\overset{\rightarrow}{5}\overset{\leftarrow}{2}\overset{\leftarrow}{1}
$,可得$b_7=2$;

$i=8$, $b_7+b_6$为奇数, 所以8的方向向右,得$
    \overset{\rightarrow}{8}\overset{\rightarrow}{3}\overset{\rightarrow}{6}\overset{\rightarrow}{7}\overset{\rightarrow}{4}\overset{\rightarrow}{5}\overset{\leftarrow}{2}\overset{\leftarrow}{1}
$,可得$b_8=0$;

于是求得中介数为$(1012120)\downarrow$, 减去2024对应$(11010)\downarrow$, 相减后后新中介数为$(1001110)\downarrow$, 可得对应的排列为47632518.

\end{document}





