\documentclass[12pt]{article}
\usepackage{amsmath} % For mathematical symbols
\usepackage{amssymb} % For more math symbols
\usepackage{geometry} % For page layout
\usepackage{ctex} % 加载 ctex 宏包以支持中文
\geometry{a4paper, margin=1in}

\title{第二章习题(基本)}

\begin{document}

\maketitle

\section*{2.1}
将集合 $\{ i \in \mathbb{Z} \mid 1 \leq i \leq 326 \}$ 划分为 5 个子集,证明必有一个子集,其中存在一个数能表示为这个子集中两个数的差。
\subsection*{解答}
反证法:设此命题不真,即存在划分
\[P_{1} \cup P_{2} \cup P_{3} \cup P_{4} \cup P_{5}=[1,326]\]
$\forall P_{i}$中不存在一个数是$P_{i}$中两数之差,$i=1,2,3,4,5$

因$\lceil\frac{326}{5}\rceil=66$,故有一子集,其中至少有66个数。
设这66个数从小到大为$a_{1},a_{2},...,a_{66}$,不妨设
$A=\{a_{1},a_{2},...,a_{66}\} \subseteq P_{1}$。

令$b_{i-1}=a_{i}-a_{1}$,$i=2,...,66$,设$B=\{b_{1},...,b_{65}\}$,$B \subseteq [1,326]$

由反证法假设$B \cap P_{1}=\emptyset$。因而
\[B \subset (P_{2} \cup P_{3} \cup P_{4} \cup P_{5})\]

因为$\lceil\frac{65}{4}\rceil=17$,不妨设$\{b_{1},...,b_{17}\} \subseteq P_{2}$。

令$c_{i-1}=b_{i}-b_{1}$,$i=1,2,...,16$,设$C=\{c_{1},...,c_{16}\}$,$C \subseteq [1,326]$

因为$c_{i-1}=b_{i}-b_{1}=(a_{i+1}-a_{1})-(a_{2}-a_{1})=a_{i+1}-a_{2}$,
由反证法假设$C \cap (P_{1} \cup P_{2}) =\emptyset$。因而
\[C \subset (P_{3} \cup P_{4} \cup P_{5})\]

因为$\lceil\frac{17}{3}\rceil=6$,不妨设$\{c_{1},...,c_{6}\} \subseteq P_{3}$。

令$d_{i-1}=c_{i}-c_{1}$,$i=1,2,...,6$,设$D=\{d_{1},...,d_{5}\}$,$D \subseteq [1,326]$

同理,由反证法假设$D \cap (P_{1} \cup P_{2} \cup P_{3}) =\emptyset$。因而
\[D \subset (P_{4} \cup P_{5})\]

因为$\lceil\frac{5}{2}\rceil=3$,不妨设$\{d_{1},d_{2},d_{3}\} \subseteq P_{4}$。

令$e_{i-1}=e_{i}-e_{1}$,$i=1,2,3$,设$E=\{e_{1},e_{2}\}$,$E \subseteq [1,326]$

同理,由反证法假设$E \cap (P_{1} \cup P_{2} \cup P_{3} \cup P_{4}) =\emptyset$。因而
\[E \subset P_{5}\]

由反证法假设$(e_{2}-e_{1}) \notin P_{5}$,且$(e_{2}-e_{1}) \notin (P_{1}\cup P_{2} \cup P_{3} \cup P_{4})$。

故$(e_{2}-e_{1}) \notin [1,326]$,显然与$(e_{2}-e_{1}) \in [1,326]$矛盾。

故得证原命题,将集合 $\{ i \in \mathbb{Z} \mid 1 \leq i \leq 326 \}$ 划分为 5 个子集,必有一个子集,其中存在一个数能表示为这个子集中两个数的差。

\section*{2.2}
设集合 $A$ 包含 13 个互不相等的实数,证明必定存在 $x, y \in A$,使得
\[0 < \frac{x - y}{1 + xy} \leq 2 - \sqrt{3}\]
\subsection*{解答}
\[\forall x \in \mathbb{R}, \exists \theta \in \left( -\frac{\pi}{2}, \frac{\pi}{2} \right), s.t. \tan{\theta} = x\]
故可以$\forall x \in A$,令$\theta_{x} = \arctan{x}$,代入原命题
\[\frac{x - y}{1 + xy} = \tan{(\theta_x - \theta_y)}\]
不妨设$x>y$,由正切函数区间单调性显然有$\theta_{x}>\theta_{y}$。

注意到$\tan{\left( \frac{\pi}{12} \right)} = 2 - \sqrt{3}$,故原命题等价于 
\[0 < \theta_x - \theta_y \leq \frac{\pi}{12} \]

将$\theta \in \left( -\frac{\pi}{2}, \frac{\pi}{2} \right)$取值范围划分为12个等长的区间,每个区间长度为$\frac{\pi}{12}$。
由于集合A包含13个互不相等的实数,所以必有至少两个实数$x>y$落入同一区间,使得
\[0 < \theta_x - \theta_y \leq \frac{\pi}{12} \]
因此得证。

\section*{2.3}
$(m+1)$ 行、$\left[ m \binom{m+1}{2} + 1 \right]$列的方格,用 $m$ 种颜色给每个方格染色,证明必能找出一个由方格组成的矩形,其四角的方格染相同颜色。
\subsection*{解答}
用$m$种颜色给$(m+1)$行方格染色,每行都至少有两个方格同色,不妨称其为每行的同色方块对。这些同色方格对在$(m+1)$ 行中一共有$\binom{m+1}{2}$种排列位置。

对于各行的方格,同色方块对(颜色,位置)的组合一共有$m \times \binom{m+1}{2}$种。

故而对于$\left[ m \binom{m+1}{2} + 1 \right]$列的方格,一定至少有两行的同色方块对颜色相同且位置相同。
这两行的同色方块对可以组成一个四角颜色相同的矩形。
因此得证。

\section*{2.4}
有 7 个互不相同的正整数,证明其中至少存在 2 个正整数 $a, b$ 使得 $a + b$ 或 $a - b$ 能被 10 除尽。
\subsection*{解答}
将所有正整数按被 10 除时的不同余数划分为 10 类,
\[[k]=\{m \mid m \bmod 10 =k\}, 0 \leq k\leq 9\]
一共可以组成6个族,分别为
\[\{[0]\}, \{[5]\}, \{[k], [10-k]\},其中1 \leq k\leq 4\]
故而 7 个不同的正整数种必有两个整数在同一族中。即其中至少存在 2 个正整数 $a, b$ 使得 $a + b$ 或 $a - b$ 能被 10 除尽。因此得证。

\section*{2.5}
设 $p \in \mathbb{Z}, q \in \mathbb{Z}^+$,证明分数 $\frac{p}{q}$ 在任意进制下均能表示为有限小数或无限循环小数的形式。
\subsection*{解答}
$z_0 = p \div q$是$\frac{p}{q}$的整数部分,$r_0 = p \mod q$是该步骤的余数。
其后的每位小数$r_{i}$在进制数$b$下的表达式为:$z_{i} = b \times r_{i-1} \div q$。
注意到:
\begin{enumerate}
    \item $b$不变的情况下,$z_{i}$完全取决于上一位的余数$r_{i-1}$。
    \item 每位余数 $r_{i} \in [0, q-1]$ 且$r_{i} \in \mathbb{N}$,因此可能互不相等的$r_{i}$共有$q$个。
\end{enumerate}

考虑前$q$步除法的余数$r_{0}, r_{1},...r_{q-1}$:
如果存在$r_{i}=0$,则$\frac{p}{q}$是一个有限小数。
如果不存在$r_{i}=0$,则由鸽巢原理,一定存在$r_{i}=r_{j}$。那么由上述两点观察,$r_{i}$和$r_{j}$之后的序列必然完全重复,则$\frac{p}{q}$是一个无限循环小数。
因此得证。

\end{document}
