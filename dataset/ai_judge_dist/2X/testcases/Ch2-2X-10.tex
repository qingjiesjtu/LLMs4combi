
\vspace{5mm} %5mm vertical space
\noindent{}\textbf{2.1}

假设存在5个子集为$A,B,C,D,E$,使得每个集合都没有数能表示成集合中两数的差,
其中至少有一个集合元素个数$\geq \lfloor\frac{326}{5}\rfloor+1=66$,设集合$A=\{a_j\}$

则对于$a_{j}-a_1(j=2,...,66)$这65个元素,都不位于$A$中,因此都位于剩下4个子集中,则至少有一个集合元素个数$\geq \lfloor\frac{65}{4}\rfloor+1=17$,设集合$B=\{b_j\}$

则对于$b_{j}-b_1(j=2,...,17)$这16个元素,都不位于$B$中,并且因为$b_{i}-b_{j}=(a_{k}-a_{1})-(a_{l}-a_{1})$,所以不在A中。因此都位于剩下3个子集中,则至少有一个集合元素个数$\geq \lfloor\frac{16}{3}\rfloor+1=6$,设集合$C=\{c_j\}$

则对于$c_{j}-c_1(j=2,...,6)$这5个元素,都不位于$A,B,C$中,因此都位于剩下2个子集中,则至少有一个集合元素个数$\geq \lfloor\frac{5}{2}\rfloor+1=3$,设集合$D=\{d_j\}$

则对于$d_2-d_1,d_3-d_2,d_3-d_1$这3个元素,不位于$A,B,C,D$中,都位于E中,但$d_2-d_1=(d_3-d_1)-(d_3-d_2)$与假设矛盾

所以存在一个集合中的数能表示成该集合中两数的差

\vspace{5mm} %5mm vertical space
\noindent{}\textbf{2.2}

区间$(-\infty,-1],(-1,0],(0,1],(1,+\infty)$中至少有一个区间包含$\geq \lfloor\frac{13}{4}\rfloor+1=4$

不妨设该4个实数在区间$(0,1]$内

假设该4个实数中${\forall} x>y$有$\frac{x-y}{1+xy} > 2-\sqrt{3}$

考虑3个区间,$A=(0,2-\sqrt{3}],B=(2-\sqrt{3},\frac{\sqrt{3}}{3}],C=(\frac{\sqrt{3}}{3},1]$,至少有两个实数位于同一个区间内

1)若这2个实数在区间A内,$0<y<x<2-\sqrt{3}$

有$\frac{x-y}{1+xy} < \frac{2-\sqrt{3}-0}{1+0} = 2-\sqrt{3}$,矛盾

2)若这两个实数在区间B内,$2-\sqrt{3}<y<x<\frac{\sqrt{3}}{3}$,令$y=x-t$

有$\frac{x-y}{1+xy} = \frac{t}{1+x(x-t)} > 2-\sqrt{3}$

则$x^2-tx+1-(2+\sqrt{3})t < 0$,

但是$x^2-tx+1-(2+\sqrt{3})t \geq 8-2\sqrt{3}-2\sqrt{3}t \geq 0 \ (when\  x=t+2-\sqrt{3}=\frac{\sqrt{3}}{3}$,矛盾


% 有$\frac{x-y}{1+xy} < \frac{\frac{\sqrt{3}}{3}-(2-\sqrt{3})}{1+(2-\sqrt{3})^2} = \frac{\sqrt{3}}{6} < 2-\sqrt{3}$,矛盾

3)若这两个实数在区间C内,$\frac{\sqrt{3}}{3}<y<x<1$,令$y=x-t$

有$\frac{x-y}{1+xy} = \frac{t}{1+x(x-t)} > 2-\sqrt{3}$

则$x^2-tx+1-(2+\sqrt{3})t < 0$,

但是$x^2-tx+1-(2+\sqrt{3})t \geq 2-(3+\sqrt{3})t \geq 0 \ (when x=t+\frac{\sqrt{3}}{3}=1)$,矛盾

综上该4个实数在区间$(0,1]$时存在两数满足该不等式。

若4个实数在$(1,+\infty)$区间内则令$x'=\frac{1}{x},y'=\frac{1}{y}$,$\frac{x-y}{1+xy}=\frac{y'-x'}{x'y'+1}$
若4个实数在$(-\infty,-1)$区间内则令$x'=-\frac{1}{x},y'=-\frac{1}{y}$,$\frac{x-y}{1+xy}=\frac{y'-x'}{x'y'+1}$
若4个实数在$(-1,0)$区间内则令$x'=y,y'=x$,$\frac{x-y}{1+xy}=\frac{y'-x'}{x'y'+1}$


\vspace{5mm} %5mm vertical space
\noindent{}\textbf{2.3}

由于每列共$m+1$行但共填入m个颜色,对于每列必有$\geq \lfloor\frac{m+1}{m}\rfloor+1=2$个方格颜色相同,这两个方格的行数记为集合${i,j}$

对于任意两行的组合,共$C_{m+1}^2$种组合,一共有$mC_{m+1}^2+1$列,因此存在$\geq \lfloor\frac{mC_{m+1}^2+1}{C_{m+1}^2}\rfloor+1=m+1$列是相同行数组合的两行颜色相同

这m+1列的同色行共有m种颜色,因此存在$\geq \lfloor\frac{m+1}{m}\rfloor+1=2$列的同色行的颜色也相同

这两列两行组成一个同色矩形。

\vspace{5mm} %5mm vertical space
\noindent{}\textbf{2.4}
集合$\{0\},\{1,9\},\{2,8\},\{3,7\},\{4,6\},\{5\}$
7个正整数中除以10的余数在同一个集合的个数$\geq \lfloor\frac{7}{6}\rfloor+1=2$个,这两个正整数满足条件


\vspace{5mm} %5mm vertical space
\noindent{}\textbf{2.5}

若p在q进制下是有限小数,则$p\cdot m^k=0\ (mod\ q)$

若p在q进制下是无限小数,则$p\cdot (m^k-1)=0\ (mod\ q)$

$\forall k, m^k$除以q的余数位于$\{0,1,...,q-1\}$,由鸽巢原理必然存在$m^i$与$m^j$余数相同$(i<j)$,则$m^j-m^i=0\ (mod\ q)$,$m^i(m^{j-i}-1)=0\ (mod\ q)$,若$m^i=0\ (mod\ q)$,则p在q进制下是有限小数;若$m^{j-i}-1=0\ (mod\ q)$,则若p在q进制下是无限小数。


\section{进阶A}


\vspace{5mm} %5mm vertical space
\noindent{}\textbf{2.6}
$a_n=10^{2n}\cdot23+a_{n-1}$

\vspace{5mm} %5mm vertical space
\noindent{}\textbf{2.7}

(1)考虑$[0,\frac{1}{n-1}),\ [\frac{1}{n-1},\frac{2}{n-1}),\ ...,[\frac{n-2}{n-1},1]$这n-1个区间,存在一个区间包含两个点,这两个点距离小于$\frac{1}{n-1}$

(2)考虑x轴的$[0,\frac{1}{\lfloor\sqrt{n}\rfloor-1}),\ [\frac{1}{\lfloor\sqrt{n}\rfloor-1},\frac{2}{\lfloor\sqrt{n}\rfloor-1}),\ ...,[\frac{\lfloor\sqrt{n}\rfloor-2}{\lfloor\sqrt{n}\rfloor-1},1]$

这$\lfloor\sqrt{n}\rfloor-1$个区间,存在一个区间包含$\geq \lfloor\frac{n}{\lfloor\sqrt{n}\rfloor-1}\rfloor+1=\lfloor\sqrt{n}\rfloor+2$个点,

同理对于y轴的$[0,\frac{1}{\lfloor\sqrt{n}\rfloor-1}),\ [\frac{1}{\lfloor\sqrt{n}\rfloor-1},\frac{2}{\lfloor\sqrt{n}\rfloor-1}),\ ...,[\frac{\lfloor\sqrt{n}\rfloor-2}{\lfloor\sqrt{n}\rfloor-1},1]$

位于x轴同一区间的$\lfloor\sqrt{n}\rfloor+2$个点,必然存在至少2个点位于y轴点同一个区间

这两个点距离小于$\frac{\sqrt{2}}{\lfloor\sqrt{n}\rfloor-1}$


\vspace{5mm} %5mm vertical space
\noindent{}\textbf{2.8}


\vspace{5mm} %5mm vertical space
\noindent{}\textbf{2.10}

除3余数相同的最多2个,否则3个余数相同的数相加和被3整除
除3余数不同时存在$0,1,2$,否则3个余数为$0,1,2$的数相加和被3整除

因此该集合最多4个元素,符合条件的集合为${1,7,3,9}$
