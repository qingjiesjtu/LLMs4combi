\documentclass{article}

\usepackage{CJKutf8}
\usepackage{setspace}


\title{第二章作业-基本}

\begin{document}
\begin{CJK}{UTF8}{gbsn}
\setstretch{1.25}
\date{}

\maketitle

\section*{2.1}
我们要证明在集合 $\{i \mid i \in \mathbb{Z}, 1 \leq i \leq 326\}$ 中,必然存在一个子集,其中至少有一个数可以表示为这个子集中两个数的差。

集合 $\{i \mid i \in \mathbb{Z}, 1 \leq i \leq 326\}$ 包含了326个连续的整数,即集合 $\{1, 2, 3, \ldots, 326\}$。

我们将这个集合划分为5个子集,根据抽屉原理,每个子集至少包含 $\lceil \frac{326}{5} \rceil = 66$ 个元素。


\begin{itemize}
    \item 326/5 上取整为 66,编号为 $a_1$-$a_{66}$。各与 $a_1$ 相减,得到 65 个差,这 65 个差分布在另外四个集合里。
    \item 65/4 上取整为 17,编号为 $b_1$-$b_{17}$。各与 $b_1$ 相减,得到 16 个差,这 16 个差分布在另外三个集合里。
    \item 16/3 上取整为 6,编号为 $c_1$-$c_6$。各与 $c_1$ 相减,得到 5 个差,这 5 个差分布在另外两个集合里。
    \item 5/2 上取整为 3,编号为 $d_1$-$d_3$。各与 $d_1$ 相减,得到 2 个差,这 2 个差分布在另外一个集合里。
\end{itemize}

这两个差记为 $e_1$ 和 $e_2$,作差 $e_2 - e_1$,这个差必然不在任何一个上述集合里(因为没有多余的集合了),产生了矛盾,所以该证明成立。

\section*{2.2}
我们要证明,对于包含 13 个互不相等的实数的集合 \( A \),必定存在 \( x, y \in A \),使得 \( 0 < \frac{x - y}{1 + xy} < 2 - \sqrt{3} \)。

首先,考虑函数 \( f(x) = \arctan(x) \),它的值域是 \( \left( -\frac{\pi}{2}, \frac{\pi}{2} \right) \)。由于集合 \( A \) 包含 13 个互不相等的实数,我们可以将这些实数通过 \( \arctan \) 函数映射到 \( \left( -\frac{\pi}{2}, \frac{\pi}{2} \right) \) 区间内。

接下来,我们将 \( \left( -\frac{\pi}{2}, \frac{\pi}{2} \right) \) 区间分成 12 个子区间,每个子区间的长度为 \( \frac{\pi}{12} \)。这些子区间是:
\[
\left( -\frac{\pi}{2}, -\frac{5\pi}{12} \right), \left( -\frac{5\pi}{12}, -\frac{\pi}{3} \right), \ldots, \left( \frac{5\pi}{12}, \frac{\pi}{2} \right)
\]

根据抽屉原理,如果我们将 13 个互不相等的实数映射到这 12 个子区间中,至少有一个子区间会包含两个数 \( x \) 和 \( y \) 的映射 \( \arctan(x) \) 和 \( \arctan(y) \)。
由于 \( \arctan(x) \) 和 \( \arctan(y) \) 在同一个长度为 \( \frac{\pi}{12} \) 的子区间内,我们有:
\[
0 < \arctan(x) - \arctan(y) \leq \frac{\pi}{12}
\]

根据反三角函数的性质,我们有:
\[
\tan(\arctan(x) - \arctan(y)) = \frac{x - y}{1 + xy}
\]

因此,我们可以得出:
\[
0 < \frac{x - y}{1 + xy} = \tan(\arctan(x) - \arctan(y)) \leq \tan\left(\frac{\pi}{12}\right) = 2 - \sqrt{3}
\]

故结论成立。
\section*{2.3}
m +1
证:每列有(m+1)行,只有 m 种颜色,故一列中必有两格同色。同色的2个格子的行号有C(m+1,2)种取法,m色共有m*C(m+1,2)同色模式。m*C(m+1, 2)+1列,根据抽屉原理,必有两列的同色组合相同。即由这两列的对应行上有4个格子同色,正好是一个矩形的4个边角
\section*{2.4}
将整数的末位数字(0~9)分成6类:{0},{5},{1,9},{2,8},{3,7},{4,6}
在所给的7个整数中,若存在两个数,其末位数字相同,则其差是10的倍数
若此7数末位数字不同,则它们中必有两个属于上述6类中的某一类,其和是10的倍数.
所以其中必有两个整数,它们的和或差是10的倍数.
\section*{2.5}
假设 \( \frac{p}{q} \) 是一个最简分数,其中 \( p \) 和 \( q \) 是互质的正整数。在任意 \( k \) 进制下,有两种情况:

1.余数出现0 ,被整除

2.余数出现0,余数序列为 \( k^n \times \frac{p}{q} \) 最多有 \( q-1 \) 个不同的值(1,2,3.... q-
1)。根据鸽巢原理,相除q次后,那么至少有两个 \( n \) 值会有相同的余数,从而形成循环节。


因此,\( \frac{p}{q} \) 在 \( k \) 进制下的小数表示是有限小数或无限循环小数。
\end{CJK}
\end{document}
