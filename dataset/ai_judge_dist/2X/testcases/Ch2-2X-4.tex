\documentclass{article}
\usepackage{graphicx} % Required for inserting images
\usepackage[utf8]{ctex}
\usepackage{titlesec}  % For customizing section titles
\usepackage{amsmath,amssymb}
\usepackage{fontspec} % Allows font customization in XeLaTeX

\usepackage[a4paper, margin=1in]{geometry} % Adjust margins here (1 inch in this case)

\title{组合数学HW2}

\begin{document}

\maketitle

\subsection*{2.1.}

反证法,假定不存在这样的子集。即存在$P1\cup P2\cup P3 \cup P4\cup P5=[1,326]$,$P_i$中不存在一个数为$P_i$中两数之差。至少有一个子集包含$\lceil\frac{326}{5}\rceil=66$个元素,不妨设为$A=\{a_1, ..., a_{66}\}\subset P_1$,其中$a_1-a_{66}$递增。\\
令$b_{i-1}=a_i-a_1$,设$B=\{b_1,...,b_{65}\}\in[1,326]$,由假设可得$B\subset P_2\cup P_3\cup p_4\cup P_5$。$\lceil\frac{65}{4}\rceil=17$,不妨设$\{b_1,...,b_{17}\}\subset P_2$,令$c_{i-1}=b_i-b_1$,设$C=\{c_1,...,c_{16}\}\in [1,326]$,由假设得$C\subset (P_3\cup P_4\cup P_5)$。 \\
又$\lceil\frac{16}{3}\rceil=6$,不妨设$\{c_1,...,c_{6}\}\subset P_3$,令$d_{i-1}=c_i-c_1$,设$D=\{d_1,...,d_5\}\in [1,326]$,由假设得$D\subset P_4\cup P_5$。
又$\lceil\frac{5}{2}\rceil=3$,不妨设$\{d_1,d_2,d_3\}\subset P_4$,令$e_{i-1}=d_i-d_1$,设$E=\{e_1,e_2\}\in [1,326]$,由假设得$E\subset P_5$,而根据假设又$e_2-e_1\notin P_5$,矛盾,故原命题得证。

\subsection*{2.2.}

对于每个实数,可以对应到一个$\theta$,满足:
\[
x_i = \tan \theta_i, \quad \theta_i \in \left(-\frac{\pi}{2}, \frac{\pi}{2}\right).
\]

原不等式可以转换为:
\[
0 < \tan(\theta_i - \theta_j) \leq 2 - \sqrt{3}.
\]

进一步可得:
\[
\theta_i - \theta_j \in \left(0, \frac{\pi}{12}\right].
\]

将$\left(-\frac{\pi}{2}, \frac{\pi}{2}\right)$均分为12个区间,因为总共有13个实数,一定有两个实数会落在同一个区间,两者之差落在$\left(0, \frac{\pi}{12}\right]$,原不等式得证。

\subsection*{2.3.}

任意选择两行总共有$\binom{m+1}{2}$种选择,颜色相同的组合共有$m\cdot \binom{m+1}{2}$个。

而总共有$m\cdot \binom{m+1}{2}+1$列,那么必有两列,其存在两行在这两列颜色相同,那么选取这两列两行所形成的矩形其四个角的颜色都相同,原命题得证。

\subsection*{2.4.}

将除10的余数$0-9$进行分类,分为$\{0\},\{5\},\{1,9\},\{2,8\},\{3,7\},\{4,6\}$共6类,7个数放进6类,必有两个数在同一类中。

若都在$\{0\}$类,那么两者相减或相加均能被10整除。

若都在$\{5\}$类,那么两者相减或相加均能被10整除。

若在剩下的类,两者相加可被10整除。

故原命题得证。

\subsection*{2.5.}

将$\frac{p}{q}$转化为任意进制$s$的过程如下:

1. $\frac{p}{q}=Iq+r_0$,其中$I$为整数部分,$r_0$是余数部分。

2. 将余数$r_0$乘以$s$,然后再除$q$,即$r_0\cdot s=d_1 q+r_1$,其中$d_1$是小数部分的第一个数字,$r_1$是新的余数。

3. 重复上述步骤。因为余数只有$0-q-1$共$q$种可能,故最多经过$q+1$步,余数就会出现重复,一旦余数开始重复,那么小数部分就开始循环。若某一步余数为0,则为有限小数。

故原命题得证。



\end{document}
